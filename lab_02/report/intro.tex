\ssr{ВВЕДЕНИЕ}

Цель данной лабораторной работы --- сравнительный анализ рекурсивного и нерекурсивного алгоритмов. Для достижения данной цели необходимо выполнить задачи:
\begin{enumerate}
    \item разработать рекурсивный и нерекурсивный алгоритмы вычисления среднего значения элементов последовательности, оканчивающейся нулём;
    \item описать средства разработки и инструменты замера процессорного времени выполнения реализации алгоритмов;
    \item реализовать разработанные алгоритмы;
    \item выполнить тестирование реализации алгоритмов;
    \item выполнить теоретическую оценку затрачиваемой реализацией каждого алгоритма памяти (для рекурсивного алгоритма --- на материале анализа высоты дерева рекурсивных вызовов и оценки затрачиваемой на один вызов функции памяти);
    \item выполнить замеры процессорного времени выполнения реализации алгоритмов в зависимости от варьируемого входа (одна точка на графике получается делением времени выполнения \textit{k} идентичных расчётов на \textit{k}, \textit{k >= 100});
    \item оценить трудоёмкость двух алгоритмов или их реализаций в худшем случае (если случаев несколько);
    \item сравнить результаты замеров процессорного времени и оценки трудоёмкости;
    \item сделать выводы из сравнительного анализа реализации рекурсивного и нерекурсивного алгоритмов решения одной и той же задачи по критериям ёмкостной эффективности (на материале теоретической оценки) и пиковой временной эффективности (на материале результатов измерений).
\end{enumerate}
