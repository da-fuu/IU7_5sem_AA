\ssr{ЗАКЛЮЧЕНИЕ}

Цель данной лабораторной работы была достигнута: проведён сравнительный анализ рекурсивного и нерекурсивного алгоритмов. Все задачи решены:
\begin{enumerate}
    \item разработаны рекурсивный и нерекурсивный алгоритмы решения задачи;
    \item описаны средства разработки и инструменты замера процессорного времени выполнения реализации алгоритмов;
    \item реализованы разработанные алгоритмы;
    \item выполнено тестирование реализации алгоритмов;
    \item выполнена теоретическая оценка затрачиваемой реализацией каждого алгоритма памяти (для рекурсивного алгоритма --- на материале анализа высоты дерева рекурсивных вызовов и оценки затрачиваемой на один вызов функции памяти);
    \item выполнены замеры процессорного времени выполнения реализации алгоритмов в зависимости от варьируемого входа;
    \item оценена трудоёмкость реализаций двух алгоритмов;
    \item проведено сравнение результатов замеров процессорного времени и оценки трудоёмкости;
    \item сделаны выводы из сравнительного анализа рекурсивного и нерекурсивного алгоритмов решения одной и той же задачи по критериям ёмкостной эффективности (на материале теоретической оценки) и пиковой временной эффективности (на материале результатов измерений) --- реализация итеративного алгоритма работает быстрее реализации рекурсивного в среднем в $3.5$ раза, использование памяти итеративным алгоритмом асимптотически оценивается как $O(1)$, а рекурсивного оценивается как $O(N)$.
\end{enumerate}
