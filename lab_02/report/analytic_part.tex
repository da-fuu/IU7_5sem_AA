\chapter{Аналитическая часть}

Рекурсия --- способ описания функций или процессов через самих себя. Рекурсивная функция --- подпрограмма, которая обращение к самой себе~\cite{recursion}. Хвостовая рекурсия --- рекурсия, при которой рекурсивный вызов является последней операцией перед возвратом из функции.
Хвостовая рекурсия позволяет компилятору выполнить оптимизацию хвостового вызова. Вместо добавления нового кадра стека для каждого рекурсивного вызова, компилятор может переиспользовать текущий кадр, эффективно превращая рекурсию в цикл.

Среднее арифметическое значение элементов последовательности чисел $x_1, x_2, \ldots, x_n$ определяется как сумма всех элементов последовательности, делённая на их количество и выражается формулой~\eqref{eq:avg_def}.

\begin{equation}
    \label{eq:avg_def}
    \bar{x} = \frac{1}{n} \sum_{i=1}^{n} x_i,
\end{equation}
где:
\begin{enumerate}
    \item $\bar{x}$ --- это среднее арифметическое значение элементов последовательности;
    \item $n$ --- число элементов в последовательности;
    \item $x_i$ --- $i$-й элемент последовательности.
\end{enumerate}

\section*{Вывод}

В аналитической части были рассмотрены определения рекурсии, рекурсивной функции, хвостовой рекурсии и среднего арифметического значения элементов последовательности чисел.
