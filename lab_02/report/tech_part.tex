\chapter{Технологическая часть}

\section{Средства реализации}

Для реализации алгоритмов был выбран язык \texttt{C}. Для измерения процессорного времени процесса используется функция \texttt{clock}~\cite{clock} из заголовочного файла \texttt{time.h}. Для построения графиков используется утилита \texttt{gnuplot}.

\section{Реализации алгоритмов}

В листинге~\ref{iter_lst} представлен исходный код реализации итеративного алгоритма вычисления среднего значения элементов последовательности.

\begin{lstlisting}[label=iter_lst, caption=Реализация итеративного алгоритма вычисления среднего значения]
double calc_avg_iter(int *array) {
  long long sum = 0;
  size_t len = 0;
  while (array[len] != 0) {
    sum += array[len];
    len++;
  }
  len++;
  double avg = (double)sum / (double)len;
  return avg;
}
\end{lstlisting}

В листинге~\ref{rec_lst} представлен исходный код реализации рекурсивного алгоритма вычисления среднего значения элементов последовательности.

\begin{lstlisting}[label=rec_lst, caption=Реализация рекурсивного алгоритма вычисления среднего значения, commentstyle=\empty]
double calc_avg_rec(int *array, long long *sum, size_t *len) {
  double avg;
  (*len)++;
  if (*array == 0) {
    avg = (double)*sum / (double)*len;
    return avg;
  }
  *sum += *array;
  array++;
  avg = calc_avg_rec(array, sum, len);
  return avg;
}
\end{lstlisting}

\section{Функциональные тесты}
В таблице~\ref{tests_table} представлены данные и результаты тестирования реализаций алгоритмов вычисления среднего значения элементов последовательности.
\begin{table}[H]
    \captionsetup{justification=raggedright,singlelinecheck=off}
    \caption{Функциональные тесты реализаций алгоритмов вычисления среднего значения элементов последовательности}
    \label{tests_table}
    \centering
    \resizebox{\textwidth}{!}{\begin{tabular}{|c|c|c|}
        \hline
        Последовательность & Ожидаемый результат & Фактический результат \\
        \hline
        $( 1, 2, 3, a )$                     &
        Сообщение об ошибке                  &
        Ошибка ввода                         \\
        \hline
        $( 0 )$                              &
        $0$                                  &
        $0$                                  \\
        \hline
        $( 1, -1, -1, 1, 2, -3, 4, -3, 0 )$  &
        $0$                                  &
        $0$                                  \\
        \hline
        $( 1, 2, 3, 3, 4, 5, 4, 5, 6, 2, 1, 7, 5, 6, 0 )$ &
        $3.6$                                &
        $3.6$                                \\
        \hline
    \end{tabular}}
\end{table}
Все тесты пройдены успешно для функций, реализующих рассматриваемые алгоритмы.

\section*{Вывод}

В технологической части определены необходимые средства реализации, и с их помощью реализованы алгоритмы вычисления среднего значения элементов последовательности. Успешно проведено функциональное тестирование реализаций алгоритмов.
