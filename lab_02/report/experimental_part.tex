\chapter{Исследовательская часть}

\section{Технические характеристики}

Замеры процессорного времени проводились на ноутбуке с характеристиками:

\begin{itemize}
    \item операционная система --- Debian Unstable GNU/Linux с ядром версии 6.16.8-1;
    \item процессор --- AMD Ryzen 5 7535HS с 6 физическими ядрами и 12 логическими ядрами с максимальной тактовой частотой 4.60~ГГц\cite{ryzen};
    \item оперативная память --- 16~Гбайт типа LPDDR5 в 4 каналах с тактовой частотой 6400~MГц и задержками CL 19 tRCD 15 tRP 17 tRAS 34 tRC 51.
\end{itemize}

\section{Замеры процессорного времени}
Замеры процессорного времени проводились для последовательностей случайных чисел, оканчивающихся нулём, длиной от $1000$ до $40000$ с шагом изменения $1000$.
Для каждого размера производилось $k = 100$ измерений, результатом замера является среднее арифметическое из этих измерений. Время измерялось в микросекундах.
Во время выполнения замеров ноутбук был подключён к сети, никаких прочих программ, кроме системных, запущено не было.

Результаты измерений представлены в таблице~\ref{table_res}.

\begin{table}[H]
    \centering
    \caption{Результаты измерений процессорного времени вычисления среднего значения элементов последовательности, мкс}
    \label{table_res}
    \begin{tabular*}{\textwidth}{@{\extracolsep{\fill}}|r|r|r|}
        \hline
        Размер & Итеративный алгоритм & Рекурсивный алгоритм \\
        \hline
        1000 & 1.606 & 4.378 \\
        \hline
        2000 & 2.686 & 8.024 \\
        \hline
        3000 & 3.334 & 10.314 \\
        \hline
        4000 & 4.300 & 13.568 \\
        \hline
        5000 & 5.166 & 16.968 \\
        \hline
        6000 & 6.172 & 20.354 \\
        \hline
        7000 & 7.326 & 24.964 \\
        \hline
        8000 & 8.110 & 28.038 \\
        \hline
        9000 & 9.118 & 31.246 \\
        \hline
        10000 & 9.994 & 34.642 \\
        \hline
        11000 & 11.360 & 39.780 \\
        \hline
        12000 & 12.046 & 42.532 \\
        \hline
        13000 & 12.776 & 44.722 \\
        \hline
        14000 & 13.716 & 47.770 \\
        \hline
        15000 & 14.662 & 52.448 \\
        \hline
        16000 & 15.526 & 54.394 \\
        \hline
        17000 & 16.600 & 58.046 \\
        \hline
        18000 & 17.392 & 61.196 \\
        \hline
        19000 & 18.498 & 64.738 \\
        \hline
        20000 & 19.268 & 68.016 \\
        \hline
        21000 & 20.372 & 71.144 \\
        \hline
        22000 & 21.096 & 75.098 \\
        \hline
        23000 & 22.090 & 78.104 \\
        \hline
        24000 & 23.250 & 80.812 \\
        \hline
        25000 & 24.404 & 85.992 \\
        \hline
        26000 & 25.186 & 89.728 \\
        \hline
        27000 & 26.004 & 92.408 \\
        \hline
        28000 & 27.002 & 95.992 \\
        \hline
        29000 & 27.932 & 99.458 \\
        \hline
        30000 & 28.840 & 102.264 \\
        \hline
        31000 & 29.896 & 105.660 \\
        \hline
        32000 & 30.936 & 108.360 \\
        \hline
        33000 & 31.626 & 111.804 \\
        \hline
        34000 & 32.660 & 114.964 \\
        \hline
        35000 & 33.742 & 118.504 \\
        \hline
        36000 & 34.680 & 121.636 \\
        \hline
        37000 & 35.376 & 125.138 \\
        \hline
        38000 & 36.484 & 128.992 \\
        \hline
        39000 & 37.368 & 131.644 \\
        \hline
        40000 & 38.290 & 136.262 \\
        \hline
    \end{tabular*}
\end{table}

Графики зависимости времени выполнения от длины последовательности приведены на рисунке~\ref{res_graph}.

\begin{figure}[H]
    \center{\includegraphics[width=16cm]{images/graph}}
    \caption{Зависимость времени выполнения от длины последовательности}
    \label{res_graph}
\end{figure}


\section*{Вывод}

В исследовательской части были произведены измерения зависимости процессорного времени работы реализации алгоритмов от длины последовательности. В результате, реализация итеративного алгоритма работает быстрее реализации рекурсивного в среднем в $3.5$ раза.
