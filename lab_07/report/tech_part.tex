\chapter{Технологическая часть}

\section{Разработка ПО с помощью Kimi K2 Thinking}

Начальный промпт предствлен на листинге~\ref{lst:prompt_kimi_1}.

\begin{lstlisting}[label=lst:prompt_kimi_1, caption=Промпт для Kimi K2 1, commentstyle=\empty]
Напиши регулярное выражение, которое в тексте, который был спаршен PyPDF2, найдет ошибку: "ВВЕДЕНИЕ (или ЗАКЛЮЧЕНИЕ) не может разбиваться на подразделы ввиду отсутствия в нём номеров", то есть проверит что в разделе ВВЕДЕНИЕ/ЗАКЛЮЧЕНИЕ (если он есть), нет подразделов. Сделай это одним регулярным выражением. В ответе напиши только выражение и краткое пояснение. Текст будет обрабатываться по страницам, поэтому необходимо просто проверить, что в тексте после подстроки ВВЕДЕНИЕ или ЗАКЛЮЧЕНИЕ, которое находится в начале страницы, нет подстроки вида числоточкачисло
\end{lstlisting}

Ответ БЯМ представлен на листинге\ref{lst:kimi_ans_1}

\begin{lstlisting}[label=lst:kimi_ans_1, caption=Ответ БЯМ Kimi K2 1, commentstyle=\empty]
    ^\s*(?:ВВЕДЕНИЕ|ЗАКЛЮЧЕНИЕ)\b[\s\S]*?\d+\.\d+
\end{lstlisting}

Второй промпт предствлен на листинге~\ref{lst:prompt_kimi_2}.

\begin{lstlisting}[label=lst:prompt_kimi_2, caption=Промпт для Kimi K2 2, commentstyle=\empty]
Теперь напиши функцию для поиска подстроки в PDF-файле с использованием разработанного регулярного выражения, возвращающую кортеж вида:
- значение, определяющее наличие строки согласно заданному регулярному выражению ("истина/ложь");
- список кортежей (если "ложь", то список пустой) с двумя составляющими каждого кортежа --- строкой, найденной с помощью регулярного выражения, и координатами для определения найденной строки в документе (минимально необходимо указать номер страницы и номер строки документа, в которой найдено вхождение искомой строки. Непосредственно таблицы и изображения строками не считать, но их названия считаются строками). Не забудь о том что нужно производить постраничную обработку
\end{lstlisting}

Ответ БЯМ представлен на листинге~\ref{lst:kimi_ans_2}

Третий промпт предствлен на листинге~\ref{lst:prompt_kimi_3}.

\begin{lstlisting}[label=lst:prompt_kimi_3, caption=Промпт для Kimi K2 3, commentstyle=\empty]
теперь реализуй программное обеспечение, принимающее на вход путь к pdf-файлу, использующее функцию из предыдущего пункта
\end{lstlisting}

Ответ БЯМ представлен на листинге~\ref{lst:kimi_ans_3}

Четвертый промпт предствлен на листинге~\ref{lst:prompt_kimi_4}.

\begin{lstlisting}[label=lst:prompt_kimi_4, caption=Промпт для Kimi K2 4, commentstyle=\empty]
исправь регулярное выражение - сделай так, чтобы слово ВВЕДЕНИЕ было не обязательно в начале страницы, достаточно только в начале строки, и при этом оно было единственным словом в строке, ищи патерн числоточкачисло только в начале строки а не где угодно
\end{lstlisting}

Ответ БЯМ представлен на листинге~\ref{lst:kimi_ans_4}

Пятый промпт предствлен на листинге~\ref{lst:prompt_kimi_5}.

\begin{lstlisting}[label=lst:prompt_kimi_5, caption=Промпт для Kimi K2 5, commentstyle=\empty]
исправь регулярное выражение - сделай так, чтобы слово ВВЕДЕНИЕ было регистронезависимым, ищи после патерна числоточкачисло обязательно большую букву после пробела
\end{lstlisting}

Ответ БЯМ представлен на листинге~\ref{lst:kimi_ans_5}


\section{Разработка ПО с помощью DeepSeek V3.2}

Начальный промпт предствлен на листинге~\ref{lst:prompt_deepseek_1}.

\begin{lstlisting}[label=lst:prompt_deepseek_1, caption=Промпт для DeepSeek V3.2 1, commentstyle=\empty]
Напиши регулярное выражение, которое в тексте, который был спаршен PyPDF2, найдет ошибку: "ВВЕДЕНИЕ (или ЗАКЛЮЧЕНИЕ) не может разбиваться на подразделы ввиду отсутствия в нём номеров", то есть проверит что в разделе ВВЕДЕНИЕ/ЗАКЛЮЧЕНИЕ (если он есть), нет подразделов. Сделай это одним регулярным выражением. В ответе напиши только выражение и краткое пояснение. Текст будет обрабатываться по страницам, поэтому необходимо просто проверить, что в тексте после подстроки ВВЕДЕНИЕ или ЗАКЛЮЧЕНИЕ (в любом регистре), которое находится в начале строки, нет подстроки вида числоточкачисло в начале строки, после которой сразу идет текст с большой буквы
\end{lstlisting}

Ответ БЯМ представлен на листинге\ref{lst:deepseek_ans_1}

\begin{lstlisting}[label=lst:deepseek_ans_1, caption=Ответ БЯМ DeepSeek V3.2 1, commentstyle=\empty]
(?im)^(?:ВВЕДЕНИЕ|ЗАКЛЮЧЕНИЕ)[\s\S]*?^\d+\.\d+\s+[А-ЯЁA-Z]
\end{lstlisting}

Второй промпт предствлен на листинге~\ref{lst:prompt_deepseek_2}.

\begin{lstlisting}[label=lst:prompt_deepseek_2, caption=Промпт для DeepSeek V3.2 2, commentstyle=\empty]
Теперь напиши функцию для поиска подстроки в PDF-файле с использованием разработанного регулярного выражения, возвращающую кортеж вида:
- значение, определяющее наличие строки согласно заданному регулярному выражению ("истина/ложь");
- список кортежей (если "ложь", то список пустой) с двумя составляющими каждого кортежа --- строкой, найденной с помощью регулярного выражения, и координатами для определения найденной строки в документе (минимально необходимо указать номер страницы и номер строки документа, в которой найдено вхождение искомой строки. Непосредственно таблицы и изображения строками не считать, но их названия считаются строками). Не забудь о том что нужно производить постраничную обработку
\end{lstlisting}

Ответ БЯМ представлен на листинге~\ref{lst:deepseek_ans_2}

Третий промпт предствлен на листинге~\ref{lst:prompt_deepseek_3}.

\begin{lstlisting}[label=lst:prompt_deepseek_3, caption=Промпт для DeepSeek V3.2 3, commentstyle=\empty]
теперь реализуй программное обеспечение, принимающее на вход путь к pdf-файлу, использующее функцию из предыдущего пункта
\end{lstlisting}

Ответ БЯМ представлен на листинге~\ref{lst:deepseek_ans_3}

Четвертый промпт предствлен на листинге~\ref{lst:prompt_deepseek_4}.

\begin{lstlisting}[label=lst:prompt_deepseek_4, caption=Промпт для DeepSeek V3.2 4, commentstyle=\empty]
сделай так, чтобы слово введение/заключение было обязано быть единственным словом в строке
\end{lstlisting}

Ответ БЯМ представлен на листинге~\ref{lst:deepseek_ans_4}

Пятый промпт предствлен на листинге~\ref{lst:prompt_deepseek_5}.

\begin{lstlisting}[label=lst:prompt_deepseek_5, caption=Промпт для DeepSeek V3.2 5, commentstyle=\empty]
проверяй что большая буква идет сразу после числа после пробела
\end{lstlisting}

Ответ БЯМ представлен на листинге~\ref{lst:deepseek_ans_5}


\section{Разработка ПО с помощью GLM-4.6}

Начальный промпт предствлен на листинге~\ref{lst:prompt_glm_1}.

\begin{lstlisting}[label=lst:prompt_glm_1, caption=Промпт для GLM-4.6 1, commentstyle=\empty]
Напиши регулярное выражение, которое в тексте, который был спаршен PyPDF2, найдет ошибку: "ВВЕДЕНИЕ (или ЗАКЛЮЧЕНИЕ) не может разбиваться на подразделы ввиду отсутствия в нём номеров", то есть проверит что в разделе ВВЕДЕНИЕ/ЗАКЛЮЧЕНИЕ (если он есть), нет подразделов. Сделай это одним регулярным выражением. В ответе напиши только выражение и краткое пояснение. Текст будет обрабатываться по страницам, поэтому необходимо просто проверить, что в тексте после подстроки ВВЕДЕНИЕ или ЗАКЛЮЧЕНИЕ (в любом регистре), которое находится в начале строки и является единственным словом в строке, нет подстроки вида числоточкачисло в начале строки, после которой сразу идет текст с большой буквы
\end{lstlisting}

Ответ БЯМ представлен на листинге\ref{lst:glm_ans_1}

\begin{lstlisting}[label=lst:glm_ans_1, caption=Ответ БЯМ GLM-4.6 1, commentstyle=\empty]
(?im)^\s*(?:ВВЕДЕНИЕ|ЗАКЛЮЧЕНИЕ)\s*$(?=[\s\S]*?^\s*\d+\.\d+\s*[А-ЯЁA-Z])
\end{lstlisting}

Второй промпт предствлен на листинге~\ref{lst:prompt_glm_2}.

\begin{lstlisting}[label=lst:prompt_glm_2, caption=Промпт для GLM-4.6 2, commentstyle=\empty]
Теперь напиши функцию для поиска подстроки в PDF-файле с использованием разработанного регулярного выражения, возвращающую кортеж вида:
- значение, определяющее наличие строки согласно заданному регулярному выражению ("истина/ложь");
- список кортежей (если "ложь", то список пустой) с двумя составляющими каждого кортежа --- строкой, найденной с помощью регулярного выражения, и координатами для определения найденной строки в документе (минимально необходимо указать номер страницы и номер строки документа, в которой найдено вхождение искомой строки. Непосредственно таблицы и изображения строками не считать, но их названия считаются строками). Не забудь о том что нужно производить постраничную обработку
\end{lstlisting}

Ответ БЯМ представлен на листинге~\ref{lst:glm_ans_2}

Третий промпт предствлен на листинге~\ref{lst:prompt_glm_3}.

\begin{lstlisting}[label=lst:prompt_glm_3, caption=Промпт для GLM-4.6 3, commentstyle=\empty]
теперь реализуй программное обеспечение, принимающее на вход путь к pdf-файлу, использующее функцию из предыдущего пункта
\end{lstlisting}

Ответ БЯМ представлен на листинге~\ref{lst:glm_ans_3}

Четвертый промпт предствлен на листинге~\ref{lst:prompt_glm_4}.

\begin{lstlisting}[label=lst:prompt_glm_4, caption=Промпт для GLM-4.6 4, commentstyle=\empty]
мне нужно использовать именно PyPDF2, переделай программу на него
\end{lstlisting}

Ответ БЯМ представлен на листинге~\ref{lst:glm_ans_4}

Пятый промпт предствлен на листинге~\ref{lst:prompt_glm_5}.

\begin{lstlisting}[label=lst:prompt_glm_5, caption=Промпт для GLM-4.6 5, commentstyle=\empty]
ты находишь ошибку там где не надо - где после 1.3 идет маленькая буква, исправь это
\end{lstlisting}

Ответ БЯМ представлен на листинге~\ref{lst:glm_ans_5}

Шестой промпт предствлен на листинге~\ref{lst:prompt_glm_6}.

\begin{lstlisting}[label=lst:prompt_glm_6, caption=Промпт для GLM-4.6 6, commentstyle=\empty]
в качестве координат ошибки возвращай координаты строчки с заголовком (число.число), а не введения
\end{lstlisting}

Ответ БЯМ представлен на листинге~\ref{lst:glm_ans_6}


\section*{Вывод}

В технологической части реализованы программы поиска ошибки <<ВВЕДЕНИЕ (или ЗАКЛЮЧЕНИЕ) не может разбиваться на подразделы ввиду отсутствия в нём номеров>> с помощью трёх БЯМ. Все реализованные программы используют только одну нестандартную библиотеку --- \texttt{PyPDF2}.
