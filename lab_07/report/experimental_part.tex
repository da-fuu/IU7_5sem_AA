\chapter{Исследовательская часть}

Результаты тестирования программы, разработанной с помощью БЯМ Kimi K2 приведены в таблице~\ref{results_kimi}.

\begin{table}[H]
	\caption{Результаты тестирования программы, разработанной с помощью БЯМ Kimi K2}
	\vspace{0.3cm}
	\begin{tabular}{|l|c|l|}
		\hline
		Название использованного pdf-файла & \parbox{4cm}{\centering Признак успешного                      \\ нахождения подстроки} & \parbox{4cm}{\centering Координаты первого\\ нахождения подстроки} \\ \hline
		\_00.pdf                           & Ложь                                      & ---                \\ \hline
		\_01.pdf                           & Ложь                                      & ---                \\ \hline
		\_02.pdf                           & Ложь                                      & ---                \\ \hline
		\_03.pdf                           & Ложь                                      & ---                \\ \hline
		\_04.pdf                           & Ложь                                      & ---                \\ \hline
		\_05.pdf                           & Ложь                                      & ---                \\ \hline
		\_06.pdf                           & Ложь                                      & ---                \\ \hline
		\_07.pdf                           & Ложь                                      & ---                \\ \hline
		\_08.pdf                           & Ложь                                      & ---                \\ \hline
		\_09.pdf                           & Ложь                                      & ---                \\ \hline
		\_10.pdf                           & Истина                                    & Стр. 3, строка 15  \\ \hline
		\_65-1.pdf                         & Истина                                    & Стр. 19, строка 18 \\ \hline
		main (2).pdf                       & Ложь                                      & ---                \\ \hline
		ВКР Селез.pdf                      & Ложь                                      & ---                \\ \hline
	\end{tabular}
	\label{results_kimi}
\end{table}

Временная сложность реализации алгоритма поиска ошибки в тексте, реализованного БЯМ Kimi K2, составляет $O(N)$, где $N$ --- длина текста.

Результаты тестирования программы, разработанной с помощью БЯМ DeepSeek V3.2 приведены в таблице~\ref{results_deepseek}.

\begin{table}[H]
	\caption{Результаты тестирования программы, разработанной с помощью БЯМ DeepSeek V3.2}
	\vspace{0.3cm}
	\begin{tabular}{|l|c|l|}
		\hline
		Название использованного pdf-файла & \parbox{4cm}{\centering Признак успешного                      \\ нахождения подстроки} & \parbox{4cm}{\centering Координаты первого\\ нахождения подстроки} \\ \hline
		\_00.pdf                           & Ложь                                      & ---                \\ \hline
		\_01.pdf                           & Ложь                                      & ---                \\ \hline
		\_02.pdf                           & Ложь                                      & ---                \\ \hline
		\_03.pdf                           & Ложь                                      & ---                \\ \hline
		\_04.pdf                           & Ложь                                      & ---                \\ \hline
		\_05.pdf                           & Ложь                                      & ---                \\ \hline
		\_06.pdf                           & Ложь                                      & ---                \\ \hline
		\_07.pdf                           & Ложь                                      & ---                \\ \hline
		\_08.pdf                           & Ложь                                      & ---                \\ \hline
		\_09.pdf                           & Ложь                                      & ---                \\ \hline
		\_10.pdf                           & Истина                                    & Стр. 3, строка 15  \\ \hline
		\_65-1.pdf                         & Истина                                    & Стр. 19, строка 18 \\ \hline
		main (2).pdf                       & Ложь                                      & ---                \\ \hline
		ВКР Селез.pdf                      & Ложь                                      & ---                \\ \hline
	\end{tabular}
	\label{results_deepseek}
\end{table}

Временная сложность реализации алгоритма поиска ошибки в тексте, реализованного БЯМ DeepSeek V3.2, составляет $O(N)$, где $N$ --- длина текста.

Результаты тестирования программы, разработанной с помощью БЯМ GLM 4.6 приведены в таблице~\ref{results_glm}.

\begin{table}[H]
	\caption{Результаты тестирования программы, разработанной с помощью БЯМ GLM 4.6}
	\vspace{0.3cm}
	\begin{tabular}{|l|c|l|}
		\hline
		Название использованного pdf-файла & \parbox{4cm}{\centering Признак успешного                      \\ нахождения подстроки} & \parbox{4cm}{\centering Координаты первого\\ нахождения подстроки} \\ \hline
		\_00.pdf                           & Ложь                                      & ---                \\ \hline
		\_01.pdf                           & Ложь                                      & ---                \\ \hline
		\_02.pdf                           & Ложь                                      & ---                \\ \hline
		\_03.pdf                           & Ложь                                      & ---                \\ \hline
		\_04.pdf                           & Ложь                                      & ---                \\ \hline
		\_05.pdf                           & Ложь                                      & ---                \\ \hline
		\_06.pdf                           & Ложь                                      & ---                \\ \hline
		\_07.pdf                           & Ложь                                      & ---                \\ \hline
		\_08.pdf                           & Ложь                                      & ---                \\ \hline
		\_09.pdf                           & Ложь                                      & ---                \\ \hline
		\_10.pdf                           & Истина                                    & Стр. 3, строка 15  \\ \hline
		\_65-1.pdf                         & Истина                                    & Стр. 19, строка 18 \\ \hline
		main (2).pdf                       & Ложь                                      & ---                \\ \hline
		ВКР Селез.pdf                      & Ложь                                      & ---                \\ \hline
	\end{tabular}
	\label{results_glm}
\end{table}

Временная сложность реализации алгоритма поиска ошибки в тексте, реализованного БЯМ GLM 4.6, составляет $O(N)$, где $N$ --- длина текста.

\section*{Вывод}

Программы, написанные разными БЯМ, имеют одинаковую временную сложность. Все они нашли в тексте все ошибки, и не произвели ложных срабатываний.
