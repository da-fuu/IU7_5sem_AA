\ssr{ВВЕДЕНИЕ}

Цель данной лабораторной работы --- разработать и реализовать программное обеспечение на языке Python для извлечения данных из текстовых файлов, полученных из PDF-файлов с применением библиотеки PyPDF2, с использованием регулярных выражений. Для достижения цели необходимо выполнить следующие задачи:

\begin{enumerate}
	\item разработать регулярные выражения для решения задачи поиска ошибки <<ВВЕДЕНИЕ (или ЗАКЛЮЧЕНИЕ) не может разбиваться на подразделы ввиду отсутствия в нём номеров>>;
	\item реализовать функцию для поиска подстроки по варианту в PDF-файле с использованием разработанных регулярных выражений, возвращающую кортеж вида:
	      \begin{itemize}
		      \item значение, определяющее наличие строки согласно заданному регулярному выражению;
		      \item список кортежей с двумя составляющими каждого кортежа --- строкой, найденной с помощью регулярного выражения, и координатами для определения найденной строки в документе.
	      \end{itemize}
	\item реализовать программное обеспечение, принимающее на вход путь к PDF-файлу, использующее функцию из предыдущего пункта;
	\item проверить реализацию на приложенных к лабораторной работе файлах, привести таблицу с колонками.
\end{enumerate}

Требуется достичь поставленной цели с использованием не менее трёх больших языковых моделей (желательно локально разворачиваемых).
