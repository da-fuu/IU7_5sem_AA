\chapter{Конструкторская часть}

\section{Процесс взаимодействия с БЯМ}
Взаимодействие со всеми выбранными БЯМ производилось с помощью официальных сайтов, предоставляющих доступ к использованию БЯМ:

\begin{enumerate}
	\item Kimi K2 Thinking --- https://www.kimi.com/;
	\item DeepSeek V3.2 --- https://chat.z.ai/;
	\item GLM-4.6 --- https://chat.deepseek.com/.
\end{enumerate}

В поле ввода вводился промпт, далее после ожидания ответа от БЯМ он проверялся на корректность. Если ответ был неверным, то модели вводился новый промпт, указывающий на то, что ответ неверен.

\section{Ограничения на поиск ошибки}

Так как в данной лабораторной работе требуется искать данную ошибку в любых pdf файлах, в том числе не удовлетворяющим ГОСТ 7.32 (в файле \texttt{\_10.pdf}, который содержит данную ошибку, следующий раздел после <<Введение>> называется <<Глава 2 Синтаксический анализатор на Prolog>>), то в качестве признака окончания раздела ВВЕДЕНИЕ и ЗАКЛЮЧЕНИЕ выбрано окончание страницы.

Разработка ПО выполнялась в 3 этапа:

\begin{enumerate}
	\item разработка регулярного выражения;
	\item реализация функции для поиска подстроки в PDF-файле с использованием разработанного регулярного выражения;
	\item реализация программного обеспечения, принимающего на вход путь к PDF-файлу, использующего функцию из предыдущего пункта.
\end{enumerate}

\section*{Вывод}

В конструкторской части был описан процесс взаимодействия с БЯМ в рамках решения задачи.
