\ssr{ВВЕДЕНИЕ}

Цель данной лабораторной работы --- разработка и сравнительный анализ последовательного и параллельного алгоритмов. Для достижения данной цели необходимо выполнить задачи:
\begin{enumerate}
	\item описать последовательный алгоритм решения задачи поиска в ориентированном графе всех вершин с количеством связей большим или меньшим, чем входной параметр (большим или меньшим --- параметр поиска);
	\item разработать параллельную версию алгоритма;
	\item реализовать обе версии алгоритма;
	\item выполнить сравнительный анализ зависимостей времени решения задач от размерности входа для реализации последовательного алгоритма и для реализации модифицированного алгоритма, запущенной с единственным вспомогательным (рабочим) потоком;
	\item выполнить сравнительный анализ зависимостей времени решения задач от размерности входа для реализации модифицированного алгоритма при $k$ вспомогательных (рабочих) потоках, $k$ принимает значения $1, 2, 4, ..., 8 \cdot q$, где $q$ --- количество логических ядер процессора ЭВМ;
	\item сформулировать рекомендацию о выборе $k$ для решения задачи на ЭВМ.
\end{enumerate}
