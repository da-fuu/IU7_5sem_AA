\chapter{Исследовательская часть}

\section{Технические характеристики}

Замеры времени проводились на ноутбуке с характеристиками:

\begin{itemize}
	\item операционная система --- Debian Unstable GNU/Linux с ядром версии 6.17.12;
	\item процессор --- AMD Ryzen 5 7535HS с 6 физическими ядрами и 12 логическими ядрами с максимальной тактовой частотой 4.60~ГГц\cite{ryzen};
	\item оперативная память --- 16~Гбайт типа LPDDR5 в 4 каналах с тактовой частотой 6400~МГц и задержками CL 19 tRCD 15 tRP 17 tRAS 34 tRC 51.
\end{itemize}

\section{Замеры времени}
Замеры времени проводились для случайных ориентированных графов размерами от $25000$ до $800000$ вершин с шагом изменения $25000$ для реализации последовательного алгоритма и для реализации параллельного алгоритма с $k$ вспомогательными потоками, $k$ принимает значения $1, 2, 4, ..., 128$.
Для каждого размера производилось $m = 100$ измерений, результатом замера является среднее арифметическое из этих измерений. Время измерялось в миллисекундах.
Во время выполнения замеров ноутбук был подключён к сети, никаких прочих программ, кроме системных, запущено не было.

Результаты измерений представлены в таблице~\ref{table_res}.

\begin{table}[H]
	\centering
	\caption{Результаты измерений времени поиска в графе вершин с необходимым количеством связей, мс}
	\label{table_res}
	\begin{tabular*}{\textwidth}{@{\extracolsep{\fill}}|r|r|r|r|r|r|r|r|r|r|}
		\hline
		\multirow{2}{*}{Размер} &
		\multirow{2}{*}{\begin{tabular}{c} Последоват. \\ алгоритм \end{tabular}} &
		\multicolumn{8}{c|}{\begin{tabular}{c} Количество вспомогательных потоков \\ в параллельном алгоритме \end{tabular}} \\
		\cline{3-10}
		& & 1 & 2 & 4 & 8 & 16 & 32 & 64 & 128 \\
		\hline
		25000 & 0.177 & 0.406 & 0.232 & 0.179 & 0.286 & 0.495 & 0.982 & 1.809 & 3.913 \\
		\hline
		50000 & 0.344 & 0.833 & 0.397 & 0.255 & 0.336 & 0.559 & 1.048 & 1.937 & 3.815 \\
		\hline
		75000 & 0.527 & 1.169 & 0.613 & 0.367 & 0.395 & 0.693 & 1.140 & 2.066 & 4.173 \\
		\hline
		100000 & 0.703 & 1.481 & 0.790 & 0.427 & 0.451 & 0.633 & 1.193 & 2.080 & 4.270 \\
		\hline
		125000 & 0.852 & 1.888 & 1.022 & 0.540 & 0.514 & 0.707 & 1.100 & 2.172 & 4.298 \\
		\hline
		150000 & 1.031 & 2.252 & 1.224 & 0.630 & 0.607 & 0.735 & 1.213 & 2.081 & 4.285 \\
		\hline
		175000 & 1.162 & 2.423 & 1.473 & 0.697 & 0.656 & 0.769 & 1.251 & 2.187 & 4.320 \\
		\hline
		200000 & 1.461 & 2.828 & 1.618 & 0.833 & 0.721 & 0.841 & 1.244 & 2.318 & 4.312 \\
		\hline
		225000 & 1.482 & 3.144 & 1.817 & 0.900 & 0.789 & 0.873 & 1.324 & 2.328 & 4.178 \\
		\hline
		250000 & 1.656 & 3.572 & 1.974 & 0.954 & 0.836 & 0.961 & 1.420 & 2.484 & 4.149 \\
		\hline
		275000 & 1.824 & 3.961 & 2.192 & 1.070 & 0.927 & 1.086 & 1.480 & 2.513 & 4.177 \\
		\hline
		300000 & 2.010 & 4.143 & 2.358 & 1.176 & 1.036 & 1.161 & 1.543 & 2.465 & 4.090 \\
		\hline
		325000 & 2.160 & 4.395 & 2.501 & 1.262 & 1.107 & 1.220 & 1.603 & 2.499 & 4.238 \\
		\hline
		350000 & 2.300 & 4.738 & 2.719 & 1.423 & 1.221 & 1.271 & 1.624 & 2.466 & 4.361 \\
		\hline
		375000 & 2.491 & 5.105 & 2.853 & 1.539 & 1.292 & 1.343 & 1.580 & 2.424 & 4.305 \\
		\hline
		400000 & 2.646 & 5.341 & 3.046 & 1.615 & 1.332 & 1.313 & 1.617 & 2.515 & 4.289 \\
		\hline
		425000 & 2.807 & 5.647 & 3.228 & 1.731 & 1.455 & 1.398 & 1.680 & 2.585 & 4.480 \\
		\hline
		450000 & 3.189 & 5.942 & 3.429 & 1.902 & 1.470 & 1.511 & 1.861 & 2.658 & 4.611 \\
		\hline
		475000 & 3.204 & 6.241 & 3.596 & 2.032 & 1.523 & 1.569 & 1.910 & 2.905 & 4.644 \\
		\hline
		500000 & 3.328 & 6.626 & 3.687 & 1.988 & 1.568 & 1.569 & 1.919 & 2.734 & 4.747 \\
		\hline
		525000 & 3.505 & 6.988 & 3.998 & 2.154 & 1.773 & 1.753 & 2.045 & 3.045 & 4.805 \\
		\hline
		550000 & 3.738 & 7.202 & 4.056 & 2.468 & 1.982 & 2.224 & 2.597 & 3.306 & 5.330 \\
		\hline
		575000 & 4.109 & 7.565 & 4.281 & 2.320 & 1.801 & 1.805 & 2.304 & 2.913 & 4.941 \\
		\hline
		600000 & 4.043 & 7.953 & 4.478 & 2.623 & 1.958 & 2.438 & 2.177 & 2.887 & 4.727 \\
		\hline
		625000 & 4.144 & 8.164 & 4.658 & 2.474 & 1.950 & 1.986 & 2.250 & 3.001 & 4.912 \\
		\hline
		650000 & 4.284 & 8.343 & 4.701 & 2.608 & 2.010 & 2.169 & 2.310 & 3.032 & 4.965 \\
		\hline
		675000 & 4.449 & 8.672 & 4.833 & 2.718 & 2.079 & 2.183 & 2.391 & 3.139 & 4.909 \\
		\hline
		700000 & 4.617 & 8.948 & 5.076 & 2.763 & 2.215 & 2.218 & 2.457 & 3.183 & 5.012 \\
		\hline
		725000 & 4.777 & 9.217 & 5.245 & 2.927 & 2.256 & 2.356 & 2.515 & 3.214 & 5.012 \\
		\hline
		750000 & 4.972 & 9.515 & 5.417 & 2.949 & 2.325 & 2.444 & 2.629 & 3.336 & 5.417 \\
		\hline
		775000 & 5.152 & 9.890 & 5.675 & 3.406 & 2.668 & 2.694 & 3.336 & 3.761 & 5.092 \\
		\hline
		800000 & 5.266 & 10.074 & 5.677 & 3.134 & 2.507 & 2.576 & 2.782 & 3.467 & 5.073 \\
		\hline
	\end{tabular*}
\end{table}

Графики зависимости времени выполнения от размера графа для реализаций последовательного и параллельного алгоритма с 1 вспомогательным потоком приведены на рисунке~\ref{res_1_graph}. Из-за накладных расходов на создание и управление потоками, а также из-за двухэтапной работы алгоритма, реализация параллельного алгоритма с 1 вспомогательным потоком работает медленнее, чем реализация последовательного.

\begin{figure}[H]
	\center{\includegraphics[width=16cm]{images/plot_001_threads}}
	\caption{Зависимость времени выполнения от размера графа для реализаций последовательного и параллельного алгоритма с 1 вспомогательным потоком}
	\label{res_1_graph}
\end{figure}


Графики зависимости времени выполнения от размера графа для реализаций последовательного и параллельного алгоритма с 2 вспомогательными потоками приведены на рисунке~\ref{res_2_graph}.

\begin{figure}[H]
	\center{\includegraphics[width=16cm]{images/plot_002_threads}}
	\caption{Зависимость времени выполнения от размера графа для реализаций последовательного и параллельного алгоритма с 2 вспомогательными потоками}
	\label{res_2_graph}
\end{figure}

Графики зависимости времени выполнения от размера графа для реализаций последовательного и параллельного алгоритма с 4 вспомогательными потоками приведены на рисунке~\ref{res_4_graph}.

\begin{figure}[H]
	\center{\includegraphics[width=16cm]{images/plot_004_threads}}
	\caption{Зависимость времени выполнения от размера графа для реализаций последовательного и параллельного алгоритма с 4 вспомогательными потоками}
	\label{res_4_graph}
\end{figure}

Графики зависимости времени выполнения от размера графа для реализаций последовательного и параллельного алгоритма с 8 вспомогательными потоками приведены на рисунке~\ref{res_8_graph}.

\begin{figure}[H]
	\center{\includegraphics[width=16cm]{images/plot_008_threads}}
	\caption{Зависимость времени выполнения от размера графа для реализаций последовательного и параллельного алгоритма с 8 вспомогательными потоками}
	\label{res_8_graph}
\end{figure}

Графики зависимости времени выполнения от размера графа для реализаций последовательного и параллельного алгоритма с 16 вспомогательными потоками приведены на рисунке~\ref{res_16_graph}.

\begin{figure}[H]
	\center{\includegraphics[width=16cm]{images/plot_016_threads}}
	\caption{Зависимость времени выполнения от размера графа для реализаций последовательного и параллельного алгоритма с 16 вспомогательными потоками}
	\label{res_16_graph}
\end{figure}

Графики зависимости времени выполнения от размера графа для реализаций последовательного и параллельного алгоритма с 32 вспомогательными потоками приведены на рисунке~\ref{res_32_graph}.

\begin{figure}[H]
	\center{\includegraphics[width=16cm]{images/plot_032_threads}}
	\caption{Зависимость времени выполнения от размера графа для реализаций последовательного и параллельного алгоритма с 32 вспомогательными потоками}
	\label{res_32_graph}
\end{figure}

Графики зависимости времени выполнения от размера графа для реализаций последовательного и параллельного алгоритма с 64 вспомогательными потоками приведены на рисунке~\ref{res_64_graph}.

\begin{figure}[H]
	\center{\includegraphics[width=16cm]{images/plot_064_threads}}
	\caption{Зависимость времени выполнения от размера графа для реализаций последовательного и параллельного алгоритма с 64 вспомогательными потоками}
	\label{res_64_graph}
\end{figure}

Графики зависимости времени выполнения от размера графа для реализаций последовательного и параллельного алгоритма с 128 вспомогательными потоками приведены на рисунке~\ref{res_128_graph}.

\begin{figure}[H]
	\center{\includegraphics[width=16cm]{images/plot_128_threads}}
	\caption{Зависимость времени выполнения от размера графа для реализаций последовательного и параллельного алгоритма с 128 вспомогательными потоками}
	\label{res_128_graph}
\end{figure}

\section*{Вывод}

В исследовательской части были произведены измерения зависимости времени работы реализаций алгоритмов от размера графа и от количества вспомогательных потоков. В результате для данной ЭВМ оптимальное значение $k$ равно $8$, это число обусловлено тем, что оно близко к числу физических ядер данной ЭВМ. При дальнейшем увеличении количества потоков время работы реализации параллельного алгоритма увеличивается из-за увеличения времени на создание и синхронизацию потоков.
