\ssr{ЗАКЛЮЧЕНИЕ}

Цель данной лабораторной работы была достигнута: были разработаны и сравнительно проанализированы последовательный и параллельный алгоритмы. Все задачи решены:
\begin{enumerate}
	\item описан последовательный алгоритм решения задачи поиска в ориентированном графе всех вершин с количеством связей большим или меньшим, чем входной параметр (большим или меньшим --- параметр поиска);
	\item разработана параллельная версия алгоритма;
	\item реализованы обе версии алгоритма;
	\item выполнен сравнительный анализ зависимостей времени решения задач от размерности входа для реализации последовательного алгоритма и для реализации модифицированного алгоритма, запущенной с единственным вспомогательным (рабочим) потоком;
	\item выполнен сравнительный анализ зависимостей времени решения задач от размерности входа для реализации модифицированного алгоритма при $k$ вспомогательных (рабочих) потоках, $k$ принимает значения $1, 2, 4, ..., 128$;
	\item сформулирована рекомендация о выборе $k$ для решения задачи на ЭВМ --- для данной ЭВМ оптимальное значение $k$ равно $8$.
\end{enumerate}
