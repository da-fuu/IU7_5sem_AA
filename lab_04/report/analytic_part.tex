\chapter{Аналитическая часть}

\section{Графы}

\textit{Граф} --- математический объект, состоящий из двух множеств. Одно из них --- любое конечное множество, его элементы называются вершинами графа. Другое множество состоит из пар вершин, эти пары называются рёбрами графа. Если множество вершин графа обозначено буквой $V$, множество рёбер --- буквой $E$, а сам граф --- буквой $G$, то граф можно обозначить как $G = (V, E)$.
Если рёбра --- упорядоченные пары, то такой граф называется \textit{ориентированным} (сокращённо орграф), если же неупорядоченные, то \textit{неориентированным}.
Ребро $(a, b)$ соединяет вершины $a$ и $b$. В графе может быть не более одного ребра, соединяющего две данные вершины. Ребро типа $(a, a)$, то есть соединяющее вершину с ней же самой, называют петлей. \textit{Степенью} вершины $v$ называется количество рёбер, инцидентных этой вершине. В случае ориентированного графа различают степень входа (количество входящих дуг) и степень выхода (количество исходящих дуг). Под количеством связей подразумевается степень вершины (сумма входящих и исходящих дуг)~\cite{graph}.

Для представления графа в памяти ЭВМ существует несколько способов:
\begin{itemize}
	\item матрица смежности --- квадратная матрица размера $|V| \times |V|$, где элемент $a_{ij}$ равен 1 (или весу ребра), если существует ребро между вершинами $i$ и $j$, и 0 в противном случае;
	\item списки смежности --- массив списков, где для каждой вершины $i$ хранится список вершин, смежных с ней.
\end{itemize}

Для разреженного графа использование матрицы смежности нецелесообразно из-за высоких затрат памяти $O(|V|^2)$. Наиболее подходящей структурой данных являются списки смежности, требующие $O(|V| + |E|)$ памяти.

Для упрощения разработки параллельного алгоритма в качестве структуры данных для представления графа были выбраны два списка смежности --- в одном хранятся вершины, соединённые с данной исходящей из неё дугой, а в другом --- вершины, дуги из которых входят в данную.

\section{Основные положения последовательного алгоритма}

Алгоритм последовательной обработки выглядит следующим образом:
\begin{enumerate}
	\item организуется цикл по всем вершинам графа от $0$ до $|V|-1$;
	\item для текущей вершины вычисляется количество смежных вершин (сумма размеров списков смежности);
	\item полученное значение сравнивается с параметром $K$ в соответствии с выбранным типом сравнения;
	\item если условие выполняется, идентификатор вершины добавляется в результирующий список;
	\item по завершении цикла возвращается список найденных вершин;
\end{enumerate}

\section{Основные положения параллельного алгоритма}

Для реализации параллельных вычислений используются следующие базовые понятия:
\begin{itemize}
	\item поток --- наименьшая единица обработки, исполнение которой может быть назначено ядром операционной системы. Потоки в рамках одного процесса разделяют общую память, но имеют собственный стек и регистровый контекст~\cite{parallel};
	\item семафор --- примитив синхронизации, представляющий собой целочисленный счётчик. Он позволяет ограничить количество потоков, одновременно выполняющих определённый участок кода, блокируя доступ при достижении счётчиком нулевого значения, а также передавать информацию между потоками~\cite{parallel}.
\end{itemize}

Поскольку обработка каждой вершины (проверка её степени) не зависит от результатов обработки других вершин, итерации основного цикла могут выполняться параллельно. Для объединения результатов работы потоков необходимо решить проблему записи в общий массив. Если организовать к нему монопольный доступ, то преимущество от параллельной обработки будет сильно уменьшено. Поэтому реализован следующий алгоритм работы:
\begin{enumerate}
	\item главный поток определяет общее количество вершин $|V|$ и количество рабочих потоков $T$;
	\item множество вершин разбивается на $T$ диапазонов. Например, поток с индексом $i$ будет обрабатывать вершины в диапазоне $[start_i, end_i)$;
	\item главный поток запускает $T$ вспомогательных потоков, передавая каждому границы его диапазона, указатель на граф и параметры поиска;
	\item каждый вспомогательный поток считает количество подходящих вершин в своём диапазоне, передаёт эту информацию главному потоку и ожидает разрешения для перехода на второй этап (блокируется на семафоре);
	\item после завершения первого этапа всеми потоками, главный поток рассчитывает смещения для записи в общем массиве для каждого потока, и разрешает потокам перейти на второй этап с помощью другого семафора;
	\item найденные вершины записываются вспомогательными потоками в общий массив по рассчитанным смещениям.
\end{enumerate}

Для синхронизации работы потоков (ожидание главным потоком завершения первого этапа, получение разрешения для перехода на второй этап) используются семафоры, так как именно это средство синхронизации служит для обмена информацией между потоками.

\section*{Вывод}
В аналитической части были рассмотрены определения графа, потока и семафора, описаны основные положения последовательного и параллельного алгоритма.
