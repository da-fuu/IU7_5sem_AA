\chapter{Конструкторская часть}

\section{Описание работы с семафором}

Взаимодействие с семафором осуществляется посредством двух основных атомарных операций:
\begin{itemize}
	\item ожидание (wait или acquire) --- операция, которая проверяет состояние счётчика. Если значение больше нуля, оно уменьшается на единицу, и поток продолжает выполнение. Если значение равно нулю, поток блокируется и переходит в режим ожидания до освобождения ресурса;
	\item сигнал (signal, post или release) --- операция, которая увеличивает значение счётчика на единицу. Если при этом существуют потоки, ожидающие на семафоре, один из них перестаёт быть заблокированным и получает доступ к ресурсу.
\end{itemize}

\section{Разработка последовательного алгоритма}

На рисунке~\ref{fig:seq_scheme} показана схема последовательного алгоритма поиска в графе вершин с количеством связей большим или меньшим, чем входной параметр.

\begin{figure}[H]
	\centerline{\includegraphics[width=12cm]{images/seq_scheme}}
	\caption{Схема последовательного алгоритма поиска в графе вершин с количеством связей большим или меньшим, чем входной параметр}
	\label{fig:seq_scheme}
\end{figure}

\section{Разработка параллельного алгоритма}

На рисунках~\ref{fig:par_main_scheme},~\ref{fig:par_main_scheme_2} и~\ref{fig:par_main_scheme_3} показана схема главного потока параллельного алгоритма поиска в графе вершин с количеством связей большим или меньшим, чем входной параметр.

\begin{figure}[H]
	\centerline{\includegraphics[width=15cm]{images/par_main_scheme}}
	\caption{Схема главного потока параллельного алгоритма поиска в графе вершин с количеством связей большим или меньшим, чем входной параметр}
	\label{fig:par_main_scheme}
\end{figure}

\begin{figure}[H]
	\centerline{\includegraphics[width=7cm]{images/par_main_scheme_2}}
	\caption{Схема главного потока параллельного алгоритма поиска в графе вершин с количеством связей большим или меньшим, чем входной параметр}
	\label{fig:par_main_scheme_2}
\end{figure}

\begin{figure}[H]
	\centerline{\includegraphics[width=7cm]{images/par_main_scheme_3}}
	\caption{Схема главного потока параллельного алгоритма поиска в графе вершин с количеством связей большим или меньшим, чем входной параметр}
	\label{fig:par_main_scheme_3}
\end{figure}

На рисунках~\ref{fig:par_help_scheme} и~\ref{fig:par_help_scheme_2} показана схема вспомогательного потока параллельного алгоритма поиска в графе вершин с количеством связей большим или меньшим, чем входной параметр.

\begin{figure}[H]
	\centerline{\includegraphics[width=15cm]{images/par_help_scheme}}
	\caption{Схема вспомогательного потока параллельного алгоритма поиска в графе вершин с количеством связей большим или меньшим, чем входной параметр}
	\label{fig:par_help_scheme}
\end{figure}

\begin{figure}[H]
	\centerline{\includegraphics[width=15cm]{images/par_help_scheme_2}}
	\caption{Схема вспомогательного потока параллельного алгоритма поиска в графе вершин с количеством связей большим или меньшим, чем входной параметр}
	\label{fig:par_help_scheme_2}
\end{figure}

\section*{Вывод}
В конструкторской части были описаны используемые структуры данных (семафоры), разработаны последовательный и параллельный алгоритмы поиска в ориентированном графе всех вершин с количеством связей большим или меньшим, чем входной параметр.
