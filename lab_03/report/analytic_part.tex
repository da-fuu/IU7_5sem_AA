\chapter{Аналитическая часть}

\section{Основные определения}

\textit{Рекурсия} --- способ описания функций или процессов через самих себя. \textit{Рекурсивная функция} --- подпрограмма, которая обращение к самой себе. \textit{Хвостовая рекурсия} --- рекурсия, при которой рекурсивный вызов является последней операцией перед возвратом из функции.
Хвостовая рекурсия позволяет компилятору выполнить оптимизацию хвостового вызова. Вместо добавления нового кадра стека для каждого рекурсивного вызова, компилятор может переиспользовать текущий кадр, эффективно превращая рекурсию в цикл~\cite{recursion}.

\textit{Граф} --- математический объект, состоящий из двух множеств. Одно из них --- любое конечное множество, его элементы называются $вершинами$ графа. Другое множество состоит из пар вершин, эти пары называются $рёбрами$ графа. Если множество вершин графа обозначено буквой $V$, множество рёбер --- буквой $E$, а сам граф --- буквой $V$, то граф можно обозначить как $G = (V, E)$.
Если рёбра --- упорядоченные пары, то такой граф называется \textit{ориентированным} (сокращённо орграф), если же неупорядоченные, то \textit{неориентированным}.
Ребро $(a, b)$ соединяет вершины $a$ и $b$. В графе может быть не более одного ребра, соединяющего две данные вершины. Ребро типа $(a, a)$, то есть соединяющее вершину с ней же самой, называют петлей~\cite{graph}.

\section{Графовые модели программ}

Между действиями программы устанавливаются два типа отношений. Если одно действие выполняется непосредственно за другим, то между двумя этими действиями установлена \textit{связь по управлению} или \textit{операционная связь}.
Если одно действие использует в качестве аргументов результаты выполнения другого действиями, то между ними установлена \textit{информационная связь}. Оба типа отношений в общем случае вводят на множестве действий частичный порядок~\cite{models}.

\textit{Граф управления} --- это ориентированный граф, построенный по исходному тексту программы следующим образом: каждой операторной конструкции программы ставится в соответствие вершина графа, между вершинами проводится направленная дуга от вершины, соответствующей предшественнику, к вершине, соответствующей последователю, если между ними установлена операционная связь. Граф управления полностью определяется структурой программы и не зависит от её входных данных: для каждой программы множества вершин и дуг фиксированы и однозначно задают единственный граф~\cite{models}.

\textit{Операционная история} --- это ориентированный граф, построенный на основе выполнения программы при заданных входных данных, в котором каждое срабатывание оператора (не обязательно уникальное) представлено отдельной вершиной, а дуги отражают непосредственную последовательность срабатываний операторов; такой граф представляет собой единственный путь от начальной вершины к конечной и полностью определяется конкретными входными данными~\cite{models}.

\textit{Информационный граф} --- это ориентированный граф, вершины которого соответствуют операторам исходной программы, а дуги отражают теоретически возможные отношения информационной связи между ними; данный граф не зависит от конкретных входных данных и представляет собой статическую модель программы, в которой могут присутствовать дуги, не реализующиеся при фактическом выполнении~\cite{models}.

\textit{Информационная история} --- это ориентированный граф, в котором каждое срабатывание оператора при выполнении программы на заданном наборе входных данных фиксируется отдельной вершиной, а дуги соединяют эти вершины в порядке передачи информации между последовательно выполняемыми операторами; такой граф представляет конкретный путь выполнения программы и полностью определяется входными данными~\cite{models}.


\section*{Вывод}

В аналитической части были рассмотрены определения рекурсии, рекурсивной функции, графов и графовых моделей алгоритмов.
