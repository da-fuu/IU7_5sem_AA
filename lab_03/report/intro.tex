\ssr{ВВЕДЕНИЕ}

Цель данной лабораторной работы --- на материале графовых моделей алгоритмов выделить участки программ, которые могут быть исполнены параллельно. Для достижения данной цели необходимо выполнить задачи:
\begin{enumerate}
    \item описать два алгоритма вычисления среднего значения элементов последовательности, окачивающейся нулём, --- рекурсивный и нерекурсивный;
    \item описать реализации двух алгоритмов четырьмя графовыми моделями --- графом управления, информационным графом, операционной историей, информационной историей;
    \item указать участки каждой программы, которые могут быть исполнены параллельно, или отсутствие таковых.
\end{enumerate}
