\chapter{Технологическая часть}

Для реализации алгоритмов был выбран язык \texttt{C}. 

\section{Реализации алгоритмов}

В листинге~\ref{iter_lst} представлен исходный код реализации итеративного алгоритма вычисления среднего значения элементов последовательности.

\begin{lstlisting}[label=iter_lst, caption=Реализация итеративного алгоритма вычисления среднего значения, numbers=left]
double calc_avg_iter(int *array) {
  long long sum = 0;
  size_t len = 0;
  while (array[len] != 0) {
    sum += array[len];
    len++;
  }
  len++;
  double avg = (double)sum / (double)len;
  return avg;
}
\end{lstlisting}

В листинге~\ref{rec_lst} представлен исходный код реализации рекурсивного алгоритма вычисления среднего значения элементов последовательности.

\begin{lstlisting}[label=rec_lst, caption=Реализация рекурсивного алгоритма вычисления среднего значения, commentstyle=\empty, numbers=left]
double calc_avg_rec(int *array, long long *sum, size_t *len) {
  double avg;
  (*len)++;
  if (*array == 0) {
    avg = (double)*sum / (double)*len;
  }
  else {
    *sum += *array;
    array++;
    avg = calc_avg_rec(array, sum, len);
  }
  return avg;
}
\end{lstlisting}

\section*{Вывод}

В технологической части определены необходимые средства реализации, и с их помощью реализованы алгоритмы вычисления среднего значения элементов последовательности.
