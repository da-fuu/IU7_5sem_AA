\chapter{Исследовательская часть}

Для реализаций алгоритмов по листингам~\ref{iter_lst} и~\ref{rec_lst} построены 4 графовые модели:
\begin{enumerate}
    \item граф управления;
    \item информационный граф;
    \item операционная история;
    \item информационная история.
\end{enumerate}

Для обозначения вершин используются номера строк из листингов.

\section{Графовые модели для реализации итеративного алгоритма}

Граф управления для реализации итеративного алгоритма приведён на рисунке~\ref{fig:iter_gu}.
\begin{figure}[H]
    \centerline{\includegraphics[height=16cm]{images/iter_gu}}
    \caption{Граф управления для реализации итеративного алгоритма}
    \label{fig:iter_gu}
\end{figure}

Информационный граф для реализации итеративного алгоритма приведён на рисунке~\ref{fig:iter_ig}.
\begin{figure}[H]
    \centerline{\includegraphics[height=18cm]{images/iter_ig}}
    \caption{Информационный граф для реализации итеративного алгоритма}
    \label{fig:iter_ig}
\end{figure}
\clearpage

Операционная история для реализации итеративного алгоритма приведена на рисунке~\ref{fig:iter_oi}.
\begin{figure}[H]
    \centerline{\includegraphics[width=15cm]{images/iter_oi}}
    \caption{Операционная история для реализации итеративного алгоритма}
    \label{fig:iter_oi}
\end{figure}
\clearpage

Информационная история для реализации итеративного алгоритма приведена на рисунке~\ref{fig:iter_ii}.
\begin{figure}[H]
    \centerline{\includegraphics[width=16cm]{images/iter_ii}}
    \caption{Информационная история для реализации итеративного алгоритма}
    \label{fig:iter_ii}
\end{figure}

Анализ графовых моделей для рекурсивного алгоритма показывает, что данная реализация алгоритма не может быть выполнена параллельно, так как каждое действие увеличения суммы (вершина 5) информационно зависит от предыдущего. 

\clearpage

\section{Графовые модели для реализации рекурсивного алгоритма}

Вершина 10 используется для обозначения рекурсивного вызова, а 10б --- для обозначения присваивания его результата переменной.

Граф управления для реализации рекурсивного алгоритма приведён на рисунке~\ref{fig:rec_gu}.
\begin{figure}[H]
    \centerline{\includegraphics[height=20cm]{images/rec_gu}}
    \caption{Граф управления для реализации рекурсивного алгоритма}
    \label{fig:rec_gu}
\end{figure}

Информационный граф для реализации рекурсивного алгоритма приведён на рисунке~\ref{fig:rec_ig}.
\begin{figure}[H]
    \centerline{\includegraphics[height=18cm]{images/rec_ig}}
    \caption{Информационный граф для реализации рекурсивного алгоритма}
    \label{fig:rec_ig}
\end{figure}
\clearpage

Операционная история для реализации рекурсивного алгоритма приведена на рисунке~\ref{fig:rec_oi}.
\begin{figure}[H]
    \centerline{\includegraphics[width=15cm]{images/rec_oi}}
    \caption{Операционная история для реализации рекурсивного алгоритма}
    \label{fig:rec_oi}
\end{figure}
\clearpage

Информационная история для реализации рекурсивного алгоритма приведена на рисунке~\ref{fig:rec_ii}.
\begin{figure}[H]
    \centerline{\includegraphics[width=17cm]{images/rec_ii}}
    \caption{Информационная история для реализации рекурсивного алгоритма}
    \label{fig:rec_ii}
\end{figure}

Анализ графовых моделей для рекурсивного алгоритма показывает, что данная реализация алгоритма не может быть выполнена параллельно, так как отсутствуют участки кода, не связанные друг с другом информационно и операционно. Так, несмотря на то, что вершина 9 не связана с вершиной 8 информационно, код вершины 8 обязательно должен быть исполнен до вершины 9, так как он использует значение переменной \texttt{array}, которое изменяется в вершине 9.

\section*{Вывод}

В исследовательской части были описаны реализации двух алгоритмов вычисления среднего значения элементов последовательности четырьмя графовыми моделями --- графом управления, информационным графом, операционной историей, информационной историей и указано отсутствие участков каждой программы, которые могут быть исполнены параллельно.

