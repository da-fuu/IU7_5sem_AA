\chapter{Исследовательская часть}

\section{Технические характеристики}

Замеры процессорного времени проводились на ноутбуке с характеристиками:

\begin{itemize}
\item операционная система --- Debian Unstable GNU/Linux с ядром версии 6.16.8-1;
\item процессор --- AMD Ryzen 5 7535HS с 6 физическими ядрами и 12 логическими ядрами с максимальной тактовой частотой 4.60~ГГц\cite{ryzen};
\item оперативная память --- 16~Гбайт типа LPDDR5 в 4 каналах с тактовой частотой 6400~MГц с таймингами CL 19 tRCD 15 tRP 17 tRAS 34 tRC 51.
\end{itemize}

\section{Замеры процессорного времени}
Замеры процессорного времени проводились для двух серий данных:
\begin{itemize}
\item квадратные матрицы размерами от 100 на 100 до 400 на 400 с шагом изменения 10;
\item квадратные матрицы размерами от 101 на 101 до 401 на 401 с шагом изменения 10.
\end{itemize}

Для каждого размера производилось 100 измерений, результатом замера является среднее арифметическое из этих измерений. Матрицы заполнялись случайными числами. Время измеряется в миллисекундах.

Во время выполнения замеров ноутбук был подключён к сети, никаких прочих программ, кроме системных, запущено не было.

Результаты измерений для первой серии данных представлены в таблице~\ref{table_even}.

Результаты измерений для второй серии данных представлены в таблице~\ref{table_odd}.
\begin{table}[H]
  \centering
  \caption{Результаты измерений для чётных размеров}
  \label{table_even}
  \begin{tabular*}{\textwidth}{@{\extracolsep{\fill}}|r|r|r|r|}
    \hline
    Размер & Стандартный алгоритм & Алгоритм Винограда & Опт. алгоритм Винограда \\
    \hline
    100 & 2.93  & 1.85  & 1.85  \\
    \hline
    110 & 3.93  & 2.49  & 2.49  \\
    \hline
    120 & 4.97  & 3.12  & 3.14  \\
    \hline
    130 & 6.30  & 3.97  & 3.98  \\
    \hline
    140 & 7.95  & 5.02  & 5.04  \\
    \hline
    150 & 9.72  & 6.19  & 6.19  \\
    \hline
    160 & 12.17 & 8.02  & 8.02  \\
    \hline
    170 & 14.21 & 9.17  & 9.15  \\
    \hline
    180 & 17.05 & 10.87 & 10.93 \\
    \hline
    190 & 19.87 & 12.58 & 12.58 \\
    \hline
    200 & 23.28 & 15.03 & 15.02 \\
    \hline
    210 & 26.86 & 17.01 & 17.01 \\
    \hline
    220 & 30.98 & 19.81 & 19.78 \\
    \hline
    230 & 35.53 & 22.61 & 22.66 \\
    \hline
    240 & 40.15 & 25.81 & 25.77 \\
    \hline
    250 & 45.79 & 28.64 & 28.64 \\
    \hline
    260 & 51.43 & 32.37 & 32.38 \\
    \hline
    270 & 57.90 & 36.20 & 36.16 \\
    \hline
    280 & 64.11 & 40.43 & 40.42 \\
    \hline
    290 & 71.77 & 45.03 & 45.04 \\
    \hline
    300 & 78.25 & 51.60 & 51.73 \\
    \hline
    310 & 87.81 & 54.79 & 54.78 \\
    \hline
    320 & 96.52 & 66.32 & 66.43 \\
    \hline
    330 & 105.15 & 65.06 & 65.11 \\
    \hline
    340 & 115.30 & 72.79 & 72.91 \\
    \hline
    350 & 126.99 & 78.67 & 78.60 \\
    \hline
    360 & 136.21 & 86.17 & 86.35 \\
    \hline
    370 & 149.56 & 93.61 & 93.66 \\
    \hline
    380 & 160.85 & 103.75 & 104.04 \\
    \hline
    390 & 174.90 & 109.09 & 109.37 \\
    \hline
    400 & 187.34 & 119.06 & 119.25 \\
    \hline
  \end{tabular*}
\end{table}

\vspace{1em}
\begin{table}[H]
  \centering
  \caption{Результаты измерений для нечётных размеров}
  \label{table_odd}
  \begin{tabular*}{\textwidth}{@{\extracolsep{\fill}}|r|r|r|r|}
    \hline
    Размер & Стандартный алгоритм & Алгоритм Винограда & Опт. алгоритм Винограда \\
    \hline
    101 & 2.97 & 1.89 & 1.90 \\
    \hline
    111 & 3.94 & 2.50 & 2.51 \\
    \hline
    121 & 5.10 & 3.23 & 3.24 \\
    \hline
    131 & 6.45 & 4.09 & 4.10 \\
    \hline
    141 & 8.09 & 5.11 & 5.13 \\
    \hline
    151 & 9.94 & 6.33 & 6.33 \\
    \hline
    161 & 12.14 & 7.66 & 7.68 \\
    \hline
    171 & 14.45 & 10.32 & 10.36 \\
    \hline
    181 & 17.15 & 11.07 & 11.10 \\
    \hline
    191 & 20.08 & 12.91 & 12.93 \\
    \hline
    201 & 23.50 & 15.55 & 15.35 \\
    \hline
    211 & 27.18 & 17.80 & 17.84 \\
    \hline
    221 & 31.41 & 20.03 & 19.93 \\
    \hline
    231 & 36.24 & 22.58 & 22.65 \\
    \hline
    241 & 41.04 & 25.97 & 25.99 \\
    \hline
    251 & 45.93 & 30.17 & 30.17 \\
    \hline
    261 & 52.09 & 33.26 & 33.28 \\
    \hline
    271 & 58.70 & 37.02 & 37.04 \\
    \hline
    281 & 65.74 & 40.94 & 40.96 \\
    \hline
    291 & 73.18 & 45.09 & 45.13 \\
    \hline
    301 & 80.78 & 51.27 & 51.50 \\
    \hline
    311 & 90.11 & 56.99 & 57.00 \\
    \hline
    321 & 100.24 & 63.97 & 64.09 \\
    \hline
    331 & 109.53 & 70.64 & 70.55 \\
    \hline
    341 & 117.81 & 79.33 & 79.35 \\
    \hline
    351 & 129.05 & 79.43 & 79.50 \\
    \hline
    361 & 140.94 & 87.74 & 87.62 \\
    \hline
    371 & 153.18 & 96.03 & 96.39 \\
    \hline
    381 & 165.31 & 101.90 & 102.02 \\
    \hline
    391 & 175.56 & 115.94 & 116.15 \\
    \hline
    401 & 190.88 & 123.81 & 123.82 \\
    \hline
  \end{tabular*}
\end{table}

Графики зависимости времени выполнения от размера матрицы для чётных размеров приведены на рисунке~\ref{even_graph}.

Графики зависимости времени выполнения от размера матрицы для нечётных размеров приведены на рисунке~\ref{odd_graph}.

\begin{figure}[h!]
	\center{\includegraphics[width=14cm]{images/benchmarks_even}}
	\caption{Зависимость времени выполнения от размера матрицы для чётных размеров}
	\label{even_graph}
\end{figure}

\begin{figure}[h!]
	\center{\includegraphics[width=14cm]{images/benchmarks_odd}}
	\caption{Зависимость времени выполнения от размера матрицы для нечётных размеров}
	\label{odd_graph}
\end{figure}

\clearpage


\section{Вывод}

В исследовательской части были произведены измерения зависимости процессорного времени работы реализации алгоритма от размера квадратной матрицы. Получены следующие результаты:

\begin{itemize}
	\item вычисление произведения матриц нечётного размера алгоритмом Винограда происходит в среднем на 1-2\% быстрее;
	\item оптимизация алгоритма Винограда не приводит к ускорению работы реализации алгоритма;
	\item реализация алгоритма Винограда работает быстрее стандартного в среднем в 1.35 раза.
\end{itemize}

\clearpage
