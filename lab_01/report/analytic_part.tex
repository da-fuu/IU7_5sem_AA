\chapter{Аналитическая часть}

\section{Основные сведения о матрицах}

В линейной алгебре матрицей называется совокупность элементов (чаще всего чисел), организованных в виде прямоугольной таблицы. Матрица, содержащая $M$ строк и $N$ столбцов, называется матрицей размера $M \times N$.

Элемент, расположенный на пересечении $i$-й строки и $j$-го столбца, обозначается как $a_{ij}$. Общий вид матрицы $A$ представляется по формуле \eqref{eq:matrix}.

\begin{equation}
	A =
	\begin{pmatrix}
		a_{11} & a_{12} & \dots  & a_{1N} \\
		a_{21} & a_{22} & \dots  & a_{2N} \\
		\vdots & \vdots & \ddots & \vdots \\
		a_{M1} & a_{M2} & \dots  & a_{MN}
	\end{pmatrix}
    \label{eq:matrix}
\end{equation}

Одной из операций, производимой с матрицами, является умножение матриц. Умножение матрицы $A$ размера $M \times P$ на матрицу $B$ размера $E \times N$ возможно только в том случае, если количество столбцов первой матрицы совпадает с количеством строк второй, то есть $P = E$. Результирующая матрица $C$ будет иметь размер $M \times N$.

\section{Стандартный алгоритм умножения матриц}

Стандартный, или классический, метод умножения матриц является прямой реализацией их математического определения~\cite{classic_alg}. Пусть даны матрица $A$ размера $M \times P$ и матрица $B$ размера $P \times N$. Их произведением будет матрица $C$ размера $M \times N$.

Каждый элемент $c_{ij}$ результирующей матрицы вычисляется как скалярное произведение $i$-й строки матрицы $A$ и $j$-го столбца матрицы $B$. Формально это выражается следующей суммой:

\begin{equation}
	c_{ij} = \sum_{k=1}^{P} a_{ik} \cdot b_{kj}
\end{equation}

Этот подход прост в реализации, однако характеризуется высокой вычислительной сложностью, которая для квадратных матриц размера $N \times N$ составляет $O(N^3)$.

\section{Алгоритм Винограда}

Алгоритм, предложенный Шмуэлем Виноградом, представляет собой метод, основанный на реорганизации вычислений с целью сокращения числа дорогостоящих операций умножения за счёт увеличения числа менее затратных операций сложения~\cite{vinograd_alg}.

Основная идея заключается в преобразовании скалярного произведения. Для этого вводятся предварительно вычисляемые величины:
\begin{itemize}
	\item факторы строк (\textit{row\_factor}): для каждой $i$-й строки матрицы $A$;
	\item факторы столбцов (\textit{col\_factor}): для каждого $j$-го столбца матрицы $B$.
\end{itemize}

Эти величины рассчитываются как суммы произведений пар соседних элементов:
\begin{equation}
	\text{row\_factor}_i = \sum_{k=1}^{\lfloor P/2 \rfloor} a_{i, 2k-1} \cdot a_{i, 2k}
\end{equation}
\begin{equation}
	\text{col\_factor}_j = \sum_{k=1}^{\lfloor P/2 \rfloor} b_{2k-1, j} \cdot b_{2k, j}
\end{equation}

Элемент $c_{ij}$ результирующей матрицы вычисляется по формуле \eqref{eq:vinograd_main}, которая содержит вдвое меньше умножений в основном цикле:
\begin{equation}
	c_{ij} = \sum_{k=1}^{\lfloor P/2 \rfloor} (a_{i, 2k-1} + b_{2k, j}) \cdot (a_{i, 2k} + b_{2k-1, j}) - \text{row\_factor}_i - \text{col\_factor}_j
	\label{eq:vinograd_main}
\end{equation}

В случае, если размерность $P$ является нечетной, к результату, полученному по формуле \eqref{eq:vinograd_main}, необходимо прибавить произведение последних элементов строки и столбца: $a_{i,P} \cdot b_{P,j}$. Предварительные вычисления факторов строк и столбцов выносятся за пределы основного тройного цикла, что и обеспечивает прирост производительности.

\section{Оптимизированный алгоритм Винограда}
\label{sec:vinograd_optimized}

При практической реализации базового алгоритма Винограда возможен ряд оптимизаций, направленных на снижение накладных расходов и более эффективное использование процессорного времени:

\begin{itemize}
	\item кэширование инвариантов цикла. Значения, которые не изменяются в ходе итераций, например, половина размера внутренней размерности ($\lfloor P/2 \rfloor$), вычисляются один раз до начала цикла. Это исключает повторное выполнение одной и той же операции;
	\item замена умножения на побитовые сдвиги. Операции умножения на 2, используемые для индексации (например, \textit{2 $\cdot$ k} и \textit{2 $\cdot$ k + 1}), могут быть заменены на более быструю операцию побитового сдвига влево (\textit{k $<<$ 1}).
\end{itemize}

Совокупность этих улучшений позволяет дополнительно ускорить выполнение алгоритма, особенно на больших размерах матриц.

\section*{Вывод}

В аналитической части были рассмотрены три подхода к умножению матриц. Стандартный алгоритм является простым в реализации, но вычислительно затратным. Алгоритм Винограда предлагает более эффективный метод за счёт сокращения числа умножений, а его оптимизированная версия представляет собой практическую доработку, направленную на снижение накладных расходов при реализации.
